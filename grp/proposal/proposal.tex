\documentclass[12pt, letterpaper, notitlepage]{article}
\usepackage[margin=1in]{geometry}  % Custom margins
\usepackage{tabularx}  % Author block table magic
\usepackage{hyperref}  % Link and mailto support
\usepackage{sectsty}
\usepackage{parskip}

\usepackage[table]{xcolor}
\sectionfont{\fontsize{16}{12}\selectfont}
\pagenumbering{gobble}

% Overwrite abstract environment
  \renewenvironment{abstract}{%
      \if@twocolumn
        \section*{\abstractname}%
      \else
        \small
        \paragraph{\abstractname:}
      \fi}
      {\if@twocolumn\else\par\bigskip\fi}

\begin{document}


% Begin title region.
\begin{center}
\vspace*{1em}
% Department/Institution
\textbf{\noindent\large Embry-Riddle Aeronautical University\\
Department of Electrical Engineering and Computer Science}

% Subtitle
\vspace{0.5em}
\textbf{\large\uppercase{Graduate Project Proposal}}

% Actual Title
\vspace{1em}
\emph{An Explainable Machine Learning Approach to Antenna Design}

% Version and editing date.
\vspace{1.5em}
\emph{
\begin{tabularx}{0.5\linewidth}{
    >{\centering\arraybackslash}X
    >{\centering\arraybackslash}X
    }
    Version: 0.0.2 &
    Date: \today
\end{tabularx}
}

% Author block. Bgroup and arraystretch fix vertical spacing issues.
\bgroup
\def\arraystretch{1.5}
\vspace{1.5em}
\emph{
\begin{tabularx}{\textwidth}{
    >{\raggedright\arraybackslash}X
    >{\centering\arraybackslash}X
    >{\raggedleft\arraybackslash}X
    }
    Course Number: SE690 & Credit Hours: 3 & Semester: Spring 2024 \\
    Author: Tyler Carr & Student ID:\ 2498305 & Student Program: MSSE \\
\end{tabularx}}
\egroup

% Email and mailto link
\vspace{1em}
\emph{Student email: \href{mailto:carrt12@my.erau.edu}{carrt12@my.erau.edu}}

\end{center}

\vspace*{\fill}

% Abstract
\abstract{Antenna simulations are a task that require a lot of resources, consume a lot of time, and are prone to interruptions. Performance metrics can only be measured for predefined antenna, geometries. By applying a machine learning algorithm to data that has already been generated from simulations of an antenna, performance metrics can be predicted significantly quicker than running full simulations. Insights about which geometric parameter had the most significant impact on the prediction can be drawn from the model and included in the output. Additionally, the model can be reversed so that for a particular form of antenna, an optimal geometry can be produced that will result in a specified performance.}

\newpage

\begin{center}
\bgroup
\footnotesize
\def\arraystretch{1.5}
\centering
\noindent\begin{tabular}{|p{12em}|l|p{12em}|l|} \hline
    \rowcolor{lightgray}& \textbf{Name} & \textbf{Signature} & \textbf{Date} \\ \hline
    \textbf{Student} & Tyler Carr & & \today \\ \hline
    \textbf{GRP Advisor} & Dr.\ Eduardo Rojas & & \today \\ \hline
    \textbf{Program Coordinator / Department Chair} & Dr.\ Masood Towhidnejad & & \today \\ \hline
\end{tabular}
\egroup
\end{center}

\newpage

\begin{center}
\centering
\noindent\begin{tabular}{|l|l|l|} \hline
    \rowcolor{lightgray}\textbf{Date} & \textbf{Version} & \textbf{Description} \\ \hline
    10/8/2023 & 0.0.1 & Initial commit, TeX Formatting\\ \hline
    10/9/2023 & 0.0.2 & Draft content for all sections\\ \hline
\end{tabular}
\vspace*{1em}
\end{center}


\section*{Objective}
The goal of the project is to develop an explainable machine learning algorithm that defines the most ideal dimensions for an antenna geometry when given specifications such as performance and dimension constraints. This algorithm aims at saving a significant amount of time instead of requiring many different simulation runs on ranges of dimensions in order to determine the ideal dimensions.\\


\section*{Problem}
Currently, simulations are performed for a variety of antenna, or metasurfaces. Depending on the type of antenna, as well as other factors, simulations can take a long time to run. The biggest factors are any interruptions to the simulation, such as power and internet outages. If a simulation is interrupted, the progress is not saved and it would have to be manually started again.

An example of a simulation whose data was provided was a simulation of an antenna with 6 varying dimensions. The data included 4,222 combinations of dimensions, each with a corresponding reflection coefficient and gain. The simulation was started on a Monday, but was interrupted by a power outage in the middle of the week. The results were finally produced on Friday of that same week.

\section*{Approach}
In order to significantly reduce the amount of time it takes to get results from simulations, a machine learning algorithm will be implemented. Once trained, this model will be able to predict what the performance is for any combination of geometric inputs to an antenna design with a particular accuracy. This prediction can be made provided that the model has already seen sufficient training data that has already been created with the simulator~\cite{Naseri_2021}.

In order to implement a machine learning model, an antenna design needs to already exist with adjustments for testable ranges for geometric parameters, frequencies, and materials. These parameters are entered into the Ansys HFSS (High Frequency Structure Simulator) program, which outputs the dB(S(1,1)), which is the reflection coefficient representing power reflected from the antenna and is ideally below -10 dB~\cite{Bevelacqua_2015}, and gain. These results are saved in the form of a CSV (comma-separated value) spreadsheet.

This data is then imported into Python as a Pandas Dataframe~\cite{reback2020pandas}. The data is preprocessed with a Sklearn (Scikit-Learn) preprocessor~\cite{scikit-learn}. This includes scaling the data, which scales each feature to a specified range, as well as removing any data rows that contain invalid data, such as a positive reflection coefficient value. After the preprocessing is complete, Sklearn's GridSearchCV is used to find the optimal parameters for the Decision Tree Regressor model that is being used This ideal model will be saved and will be used to predict the reflection coefficient and gain for any future geometric parameters corresponding to the same antenna type.

SHAP (Shapley Additive Explanations) is then applied to the trained machine learning model~\cite{lundberg2017unified}. This is used to determine which geometric parameters have the highest impact on the predictions of the model. If a certain parameter has a significantly higher impact than the others, it will be given higher preference when choosing the parameters for the creation of an antenna with that particular set of parameters.

The end goal will be creating a program that can take reflection coefficient, gain, and geometric constraints as inputs and determine the best geometry for an antenna for those inputs. Since the model currently takes the geometry as an input and outputs the performance, this will involve making the model work backwards by reversing the inputs and outputs. A GUI (graphical user interface) will be created where the user will be able to enter the inputs and view the output in user-friendly format. 

\section*{Deliverables and Timeline}
\begin{center}
\centering
Table: Project deliverables and timeline\\
\noindent\begin{tabular}{|l|l|l|} \hline
  \rowcolor{lightgray}\textbf{\#} & \textbf{Deliverables} & \textbf{Date} \\ \hline
  1 & Problem Description & 2/10/2024\\ \hline
  2 & Completed machine learning model & 3/22/2024\\ \hline
  3 & GUI to choose antenna parameters based on desired performance & 4/12/2024\\ \hline
  4 & GRP Report & 4/26/2024\\ \hline
  5 & GRP Presentation & 5/1/2024\\ \hline
\end{tabular}
\end{center}
\hspace{1em}

\bibliographystyle{unsrt}
\bibliography{refs}


\end{document}